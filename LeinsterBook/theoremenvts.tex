% Basic Category Theory
% Tom Leinster <Tom.Leinster@ed.ac.uk>
% 
% Copyright (c) Tom Leinster 2014-2016
% 
% Settings for theorem-like environments
% 


% Package and general parameters:


\usepackage[amsmath,thmmarks]{ntheorem}

\theoremseparator{\hspace{0.5em}}       % This is the symbol between the
                                % header (e.g. "Theorem") and statement of
                                % a theorem.


% Theorem-like environments with bodies set in italics:


\newtheorem{thm}{Theorem}[section]
\newtheorem{propn}[thm]{Proposition}
\newtheorem{lemma}[thm]{Lemma}
\newtheorem{cor}[thm]{Corollary}

\newtheorem{ilemma}{Lemma}[chapter]     % For use in introduction
\newtheorem{alemma}{Lemma}[chapter]     % For use in appendix


% Theorem-like environments with bodies set in upright text:


\theorembodyfont{\normalfont}

\newtheorem{defn}[thm]{Definition}
\newtheorem{example}[thm]{Example}
\newtheorem{examples}[thm]{Examples}
\newtheorem{remark}[thm]{Remark}
\newtheorem{remarks}[thm]{Remarks}
\newtheorem{warning}[thm]{Warning}
\newtheorem{notn}[thm]{Notation}
\newtheorem{constn}[thm]{Construction}
\newtheorem{question}[thm]{\hspace{-.5ex}}      % Exercise

\newtheorem{iexample}[ilemma]{Example}          % For use in intro
\newtheorem{iquestion}[ilemma]{\hspace{-.5ex}}  % For use in intro
\newtheorem{aquestion}[alemma]{\hspace{-.5ex}}  % For use in appendix


% Proof environments


\theoremstyle{nonumberplain}
\theoremsymbol{\ensuremath{\Box}}
\qedsymbol{\ensuremath{\Box}}

\newtheorem{pf}{Proof}

% The following commands are for proofs coming significantly later than the
% statement
\newcommand{\theoremtobeproved}{}
\newtheorem{pfoftheorem}{Proof of \theoremtobeproved}
\newenvironment{pfof}[1]
{
\renewcommand{\theoremtobeproved}{#1}
\begin{pfoftheorem}
}
{\end{pfoftheorem}}





